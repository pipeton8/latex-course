% Consideremos el código del teorema anterior e intentemos incorporarle una portada
\documentclass{article}

% Paquetes %
\usepackage[utf8]{inputenc}
\usepackage[spanish]{babel}
\usepackage{mathtools, amsfonts, amsmath, amsthm, amssymb}
\usepackage{setspace}

% Para esta portada vamos a poner una cita, para darle un poco de ``estilo''.
% El paquete encargado de hacer esto será epigraph.
\usepackage{epigraph}
\setlength{\epigraphwidth}{0.35\textwidth} % Con este comando personalizamos el ancho del cuadro de la cita.

% Márgenes y espaciado %
\usepackage[margin=2cm]{geometry}
\spacing{1.2}

% Entornos %
\newtheorem{teo}{Teorema}

% Esta portada es parecida a la portada simple, pero incorpora otros elementos típicos de una tesis.
% Primero debemos decirle a LaTeX cuál es el título, quién es el autor y, opcionalmente, cuál es la fecha. Si esta última no se especifica, LaTeX asume la fecha del día en que se compila el texto.
\author{Felipe Del Canto M.\footnote{Con el comando \textbackslash\texttt{footnote} se incorpora una nota al pie, donde se pueden indicar afiliaciones, pequeños agradecimientos y otros mensajes.}}
\title{Aquí va el título \\ Usando el comando \texttt{\textbackslash\textbackslash} se puede poner un subtítulo}
\date{Con el comando \textbackslash\texttt{date} se puede poner la fecha}

% Documento %
\begin{document}

% Ahora, debemos decirle a LaTeX que haga la portada. Para ello usaremos el entorno titlepage, que deja
% la portada en una página aparte, y el comando \maketitle, que usa los datos entregados anteriormente y
% los usa para crear la portada.
\begin{titlepage}
\thispagestyle{empty}
\maketitle

% Ahora pondremos la cita, para que esté bajo el título y antes del abstract.
% La cita se incorpora con el comando \epigraph, que acepta 2 argumentos:
%	-	El primero es la cita misma
%	-	El segundo es la referencia
%
% En el texto, entre ambos argumentos aparece una linea separatoria. El comando \vfill ayuda a que la posición vertical de estos elementos se vea agradable, no muy pegado al título ni al final de la página.
\vfill
\epigraph{Perdimos porque no ganamos}{\textsc{Ronaldo}}

% Para este ejemplo también incorporaremos un abstract, explicando brevemente nuestro trabajo y sus hallazgos. Para asegurarse que LaTeX entiende qué es lo que va en el abstract, podemos encerrar el comando y el texto entre {}.
\vfill
{\abstract En este trabajo aprendí a escribir una portada. Todavía no es la mejor portada de la vida pero me permite escribir un texto del que me sienta orgulloso.}
\vfill

\end{titlepage}

\begin{teo} Demuestre que si $1 \leq p \leq q \leq \infty$, entonces
	$$\|x\|_{\ell^{q}} \leq \|x\|_{\ell^{p}}$$
	
En particular $\ell^{p} \subset \ell^{q}$.
\end{teo}

\begin{proof} El caso $q = \infty$ es claro:
	$$\|x\|_{\ell^{\infty}} 
		= 	\sup_{j \in \mathbb{N}} |x_{j}|
		= 	\left(\sup_{j \in \mathbb{N}} |x_{j}|^{p}\right)^{1/p} 
		\leq	\left(\sum_{j \in \mathbb{N}} |x_{j}|^{p}\right)^{1/p} = \|x\|_{\ell^{p}}
	$$

Además, si $p = q$ el resultado es directo. Consideremos entonces $1 \leq p < q < \infty$, tenemos que
	\begin{align*}
		\|x\|_{\ell^{q}}^{q}
			&=		\sum_{j \in \mathbb{N}} |x_{j}|^{q}	\\
			&=		\sum_{j \in \mathbb{N}} |x_{j}|^{p} |x_{j}|^{q-p}	\\
			&\leq		\|x\|^{q-p}_{\ell^{\infty}} \sum_{j \in \mathbb{N}} |x_{j}|^{p}	\\
			&=		\|x\|^{q-p}_{\ell^{\infty}} \|x\|_{\ell^{p}}^{p}	\\
			&\leq		\|x\|^{q-p}_{\ell^{p}} \|x\|_{\ell^{p}}^{p}	\\
			&=		\|x\|_{\ell^{p}}^{q}		 
	\end{align*}

De donde se obtiene el resultado.
\end{proof}                                          

\end{document}

