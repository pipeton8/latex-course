% Consideremos el código del teorema anterior con la portada y mejorémoslo
\documentclass{article}

% Paquetes %
\usepackage[utf8]{inputenc} 
\usepackage[spanish]{babel}
\usepackage{mathtools, amsfonts, amsmath, amsthm, amssymb}
\usepackage{setspace}
\usepackage{graphicx}

% Márgenes y espaciado %
\usepackage[margin=2cm]{geometry}
\spacing{1.2}

% Hay veces que escribir demasiadas veces un comando largo puede ser tedioso y además puede ser lento hacerlo. Por ejemplo, para escribir los números reales, escribir \mathbb{N} toma tiempo, sobre todo si uno se equivoca. Para ello nos gustaría escribir una función, que se llame de manera más corta. La forma de crear funciones es la siguiente
%	\newcommand{\nuevonombre}{que hace la funcion}
 
% En el caso del ejemplo de \mathbb{N} podemos crear la siguiente función
\newcommand{\N}{\mathbb{N}}

% Ahora, cada vez que escribamos \N, LaTeX entiende que estamos escribiendo \mathbb{N}. Observar que es como que abreviáramos la escritura, por lo que sigue siendo necesario estar en modo matemático para usar esta función.

% También es posible crear funciones que acepten argumentos, para crearlas se usa el siguiente formato
%	\newcommand{\nombrefuncion}[numerodeargumentos]{que hace la función}

% En la parte donde definimos que hace la función podemos referenciar los argumentos por ``#n'', donde n es el número de argumento. Veamos el siguiente ejemplo:
\newcommand{\twodef}[4]{\left\{ \begin{aligned}
						&#1	&&\text{si } #2	\\
						&#3	&&\text{si } #4
					\end{aligned} \right. }

% Esta función permite escribir funciones de dos ramas. Como podemos ver, recibe 4 argumentos, el primero es la primera rama de la función, el segundo, la condición de esa rama. El tercero y el cuarto se definen de manera similar. Para llamar a esta función en el texto tendremos que escribir
%	\twodef{primera rama}{condicion primera rama}{segunda rama}{condicion segunda rama}

% Por un tema de orden del código y mejor entendimiento de lo que escribimos, también podemos escribir
%	\twodef	{primera rama}{condicion primera rama}
%			{segunda rama}{condicion segunda rama}

% Esta manera de escribir imita lo que uno ve en el resultado y tiene una lectura mucho mejor. 

% Vale la pena notar que el entorno aligned es equivalente al align, pero este nuevo entorno puede combinarse en modo matemático (es decir, para cosas entre $$) de manera que se pueda complementar con más texto.
					
% Entornos %
\newtheorem{teo}{Teorema}

% Queremos probar la función que creamos después de la demostración del teorema y lo haremos en un nuevo entorno, uno de una observación. Nos gustaría que la observación se enumerara según el teorema, para ello debemos decirle a LaTeX que lo haga. Esto lo logramos de la siguiente manera
\newtheorem{obst}{Observación}[teo]

% La primera parte, \newtheorem{obst}{Observación}, es la misma sintaxis para el teorema, solo que ahora queremos que diga ``Observación'' y no ``Teorema''. La parte final, [teo], le dice a LaTeX que queremos que este entorno siga la numeración del teorema.

% Documento %
\begin{document}

\begin{titlepage}
	\begin{minipage}{2.5cm}
		\includegraphics[width=2cm]{logouccolor.png}
	\end{minipage}
	\begin{minipage}{13 cm}
		\begin{flushleft}
   			\noindent\large{\sc
				Pontificia Universidad Católica de Chile \\ 
		     		Facultad de Matemáticas \\ 
		     		Departamento de Matemáticas \\ 
				Segundo semestre 2017
		     	}
		\end{flushleft}
	\end{minipage}
	
	% Título %
\begin{center}
	\vspace*{\fill}
		\Huge\textbf{Primera portada}	\\
		\Huge\textbf{taller de \LaTeX}	\\
		\LARGE{Yo mismo} \\
	\vspace*{\fill}
		
	\vfill
		
	\Large{Profesor: No importa por ahora}
\end{center}
\end{titlepage}

\begin{teo} Demuestre que si $1 \leq p \leq q \leq \infty$, entonces
	$$\|x\|_{\ell^{q}} \leq \|x\|_{\ell^{p}}$$
	
En particular $\ell^{p} \subset \ell^{q}$.
\end{teo}

\begin{proof} El caso $q = \infty$ es claro:
	$$\|x\|_{\ell^{\infty}} 
		= 	\sup_{j \in \N} |x_{j}|
		= 	\left(\sup_{j \in \N} |x_{j}|^{p}\right)^{1/p} 
		\leq	\left(\sum_{j \in \N} |x_{j}|^{p}\right)^{1/p} = \|x\|_{\ell^{p}}
	$$

Además, si $p = q$ el resultado es directo. Consideremos entonces $1 \leq p < q < \infty$, tenemos que
	\begin{align*}
		\|x\|_{\ell^{q}}^{q}
			&=		\sum_{j \in \N} |x_{j}|^{q}	\\
			&=		\sum_{j \in \N} |x_{j}|^{p} |x_{j}|^{q-p}	\\
			&\leq		\|x\|^{q-p}_{\ell^{\infty}} \sum_{j \in \N} |x_{j}|^{p}	\\
			&=		\|x\|^{q-p}_{\ell^{\infty}} \|x\|_{\ell^{p}}^{p}	\\
			&\leq		\|x\|^{q-p}_{\ell^{p}} \|x\|_{\ell^{p}}^{p}	\\
			&=		\|x\|_{\ell^{p}}^{q}		 
	\end{align*}

De donde se obtiene el resultado.
\end{proof}                                          

% Ahora usaremos la función de muchos argumentos creada anteriormente para complementar la información del teorema. Usamos el nuevo entorno creado
\begin{obst} Hemos visto que $x = \{x_{n}\}_{n \in \N}$ definida por
	$$x_{n} = \twodef	{\frac{1}{n-1}} {n > 1}
					{0}{n = 1}$$

pertenece a $\ell^{2}$, por lo que por el teorema anterior, pertenece a $\ell^{q}$ para todo $q \geq 2$.
\end{obst}
\end{document}

