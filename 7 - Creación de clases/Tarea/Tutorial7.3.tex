\RequirePackage[patch]{kvoptions}

% Tipo de documento %
\documentclass[duedate = 11 de Septiembre, 
			ramo = An\'alisis Funcional, 
			doctype = Tarea 1,
			semester = 2,
			year = 2017]{tarea}

% Paquetes %
\usepackage{tutorialLaTeX}

%%% Documento %%%
\begin{document}

% Portada %
\begin{titlepage} 
\dotitlepage
\end{titlepage}


% Contenido %
\pagestyle{empty}

%P1%
\section*{Problema 1}
Sea $X := C^{\infty}(\R)$. Definamos
	\begin{align*}
		\|x\|_{j,k}	
			&:=	\max_{i \in [-j,j]} |x^{(k)}(t)|, \qquad x \in X, \, j \in \N, \, k \in \Z_{+}	\\
		d(x,y)
			&:=	\sum_{j=1}^{\infty} \sum_{k=0}^{\infty} 2^{-j-k} \frac{\|x-y\|_{j,k}}{1 + \|x-y\|_{j,k}}, \qquad x,y \in X 
	\end{align*}

Demuestre que $(X,d)$ es espacio métrico completo.

\begin{proof}[Solución] Observar que $d$ está bien definida pues
	$$\frac{\|x-y\|_{j,k}}{1 + \|x-y\|_{j,k}} \leq 1 \qquad \paratodo j \in \N, \, k \in \Z_{+}$$

de manera que, para todo $x,y \in X$,
	\[
		|d(x,y)|
			=		d(x,y)
			\leq		\sum_{j=1}^{\infty} \sum_{k=0}^{\infty} 2^{-j-k}
			=		\sum_{j=1}^{\infty} 2^{-j} \left(\sum_{k=0}^{\infty} 2^{-k}\right)
			=		2\sum_{j=1}^{\infty} 2^{-j}
			<		\infty
	\tag{1.1}\]

Ahora debemos verificar que es métrica. La simetría es directa de la misma propiedad sobre $\|\cdot\|_{j,k}$ para todo $j \in \N, k \in \Z_{+}$ (porque son semi normas si $k \neq 0$ y normas si $k = 0$). Observar que $d(x,y) = 0$ implica, como todos los términos de la serie son positivos, que
	$$\|x-y\|_{j,k} = 0,	\qquad \paratodo j \in \N, \, k \in \Z_{+}$$

En particular, $\|x-y\|_{j,0} =0 $ y, como es norma, entonces $x\big|_{[-j,j]} = y\big|_{[-j,j]}$ para cada $j \in \N$, luego $x = y$. Que $d(x,x) = 0$ es directo pues $\|x-x\|_{j,k} = 0$ para todo $j \in \N, k \in \Z_{+}$. Para la desigualdad triangular, notemos que
	$$\func{f}{[0,\infty)}{\R}
			{x}{\frac{x}{1+x} = 1 - \frac{1}{1+x}}$$

es creciente pues $f'(x) = \frac{1}{(1+x)^{2}} > 0$. Como para cada $j \in \N$, $k \in \Z_{+}$, $x,y,z \in X$,
	$$0 \leq \|x-y\|_{j,k} \leq \|x-z\|_{j,k} + \|z-y\|_{j,k}$$

entonces usando la monotonía de $f$:
	\begin{align*}
		\frac{\|x-y\|_{j,k}}{1 + \|x-y\|_{j,k}}	
			\leq		\frac{\|x-z\|_{j,k} + \|z-y\|_{j,k}}{1+\|x-z\|_{j,k} + \|z-y\|_{j,k}}
			\leq		\frac{\|x-z\|_{j,k}}{1+\|x-z\|_{j,k}} + \frac{\|z-y\|_{j,k}}{1 + \|z-y\|_{j,k}} 
	\end{align*}

De donde se obtiene directamente que para cada $x,y,z \in X$,
	$$d(x,y) \leq d(x,z) + d(y,z)$$

Para la completitud, sea $\{x_{n}\}_{n \in \N} \subset X$ sucesión de Cauchy. Sin pérdida de generalidad supongamos que las $x_{n}$ son reales. Primero busquemos el candidato a límite. Sean $t \in \R$, $k \in \Z_{+}$ y $j \in \N$ con $t \in [-j,j]$. Dado $\epsilon > 0$, como $\{x_{n}\}_{n \in \N}$ es de Cauchy, existe $N \in \N$ tal que $m,n \geq N$ implica
	$$2^{-j-k} \frac{\|x_{n} - x_{m}\|_{j,k}}{1 + \|x_{n} - x_{m}\|_{j,k}} \leq d(x_{n}, x_{m}) < 2^{-j-k} \frac{\epsilon}{1+\epsilon}$$

Luego
	\begin{align*}
		\frac{\|x_{n} - x_{m}\|_{j,k}}{1 + \|x_{n} - x_{m}\|_{j,k}}
			&<	\frac{\epsilon}{1+\epsilon}	\\
		\|x_{n} - x_{m}\|_{j,k} + \epsilon \|x_{n} - x_{m}\|_{j,k}
			&<	\epsilon \|x_{n} - x_{m}\|_{j,k} + \epsilon	\\
		\|x_{n} - x_{m}\|_{j,k}
			&<	\epsilon
	\end{align*}
	
Y así
	$$|x_{n}^{(k)}(t) - x_{m}^{(k)}(t)| \leq \max_{s \in [-j,j]} |x_{n}^{(k)}(s) - x_{m}^{(k)}(s)| < \epsilon$$

Lo que dice que $\{x_{n}^{(k)}(t)\}_{t \in \R}$ es de Cauchy en $\R$ y por lo tanto convergente a $\widetilde{x}_{k}(t)$. Observar más aún que la convergencia es uniforme en cualquier intervalo $[-j,j]$, de manera que como $x_{n}^{(k)}$ son continuas, entonces $\widetilde{x}_{k}: t \mapsto \widetilde{x}_{k}(t)$ también es continua (pues cada $t \in \R$ se encuentra en el interior de en algún intervalo de la forma $[-j,j]$, con $j \in \N$). Definamos $x(t) := \widetilde{x}_{0}(t)$ para $t \in \R$, la cual es continua por la observación anterior. Queremos mostrar que esta función es efectivamente el límite.
	\begin{afirmacion}
		\item $x \in C^{\infty}(\R)$.
		
		\begin{proof} Basta probar que $x^{(k)}$ existe y coincide con $\widetilde{x}_{k}$ para todo $t \in \R$ y $k \geq 1$. Probemos esto por inducción sobre $k$. Sea $k = 1$, dado $t \in \R$, con $t \in (-j,j)$ para algún $j \geq 1$ tenemos que
			\begin{align*}
				x_{n}		&\tiende{n \to \infty} x			\quad	\text{ uniformemente en } [-j,j]	\\
				x'_{n}	&\tiende{n \to \infty} \widetilde{x}_{1}	\quad	\text{ uniformemente en } [-j,j]
			\end{align*}
		
		Por teorema fundamental del cálculo,
			$$x_{n}(t) - x_{n}(-j) = \int_{-j}^{t} x'_{n}(s) \, ds$$
		
		Por convergencia (uniforme),
			\begin{align*}
				\int_{-j}^{t} x'_{n}(s) \, ds 
					&\tiende{n \to \infty}	\int_{-j}^{t} \widetilde{x}_{1}(s) \, ds	\\
				x_{n}(t) - x_{n}(-j)
					&\tiende{n \to \infty}	x(t) - x(-j)
			\end{align*}
		
		Luego,
			$$x(t) - x(-j) = \int_{-j}^{t} \widetilde{x}_{1}(s) \, ds $$
		
		Por teorema fundamental del cálculo, como $x_{1}$ es continua, la función de la derecha (y por lo tanto también la de la izquierda) es derivable y
			$$x'(t) = (x(t) - x(-j))' = \left(\int_{-j}^{t} \widetilde{x}_{1}(s) \, ds\right)' = \widetilde{x}_{1}(t) $$ 
		
		mostrando que $x'(t)$ existe en cada punto $t \in \R$ (nuevamente, cada $t \in \R$ pertenece a un intervalo $(-j,j)$ para cierto $j \in \N$) y, más aún $x' \equiv \widetilde{x}_{1}$, en particular $x'$ es continua pues $\widetilde{x}_{1}$ lo es.	\\
		
		Ahora, supongamos que demostramos que $x^{(k)}$ existe y coincide con $\widetilde{x}_{k}$. Nuevamente, dado $t \in \R$, con $t \in (-j,j)$ para algún $j \geq 1$, tenemos que
			\begin{align*}
				x^{(k)}_{n}		&\tiende{n \to \infty} \widetilde{x}_{k} = x^{(k)}	\quad	\text{ uniformemente en } [-j,j]	\\
				x^{(k+1)}_{n}	&\tiende{n \to \infty} \widetilde{x}_{k+1}		\quad	\text{ uniformemente en } [-j,j]
			\end{align*}

		Por teorema fundamental del cálculo,
			$$x^{(k)}_{n}(t) - x^{(k)}_{n}(-j) = \int_{-j}^{t} x^{(k+1)}_{n}(s) \, ds$$
		
		Por convergencia (uniforme),
			\begin{align*}
				\int_{-j}^{t} x^{(k+1)}_{n}(s) \, ds 
					&\tiende{n \to \infty}	\int_{-j}^{t} \widetilde{x}_{k+1}(s) \, ds	\\
				x^{(k)}_{n}(t) - x^{(k)}_{n}(-j)
					&\tiende{n \to \infty}	x^{(k)}(t) - x^{(k)}(-j)
			\end{align*}

		Luego,
			$$x^{(k)}(t) - x^{(k)}(-j) = \int_{-j}^{t} \widetilde{x}_{k+1}(s) \, ds $$

\newpage		
		Por teorema fundamental del cálculo, como $x_{k+1}$ es continua, la función de la derecha (y por lo tanto también la de la izquierda) es derivable y
			$$x^{(k+1)}(t) = (x^{(k)}(t) - x^{(k)}(-j))' = \left(\int_{-j}^{t} \widetilde{x}_{k+1}(s) \, ds\right)' = \widetilde{x}_{k+1}(t) $$ 

		mostrando que $x^{(k+1)}(t)$ existe en cada punto $t \in \R$ (nuevamente, cada $t \in \R$ pertenece a un intervalo $(-j,j)$ para cierto $j \in \N$) y, más aún $x^{(k+1)} \equiv \widetilde{x}_{k+1}$, en particular $x^{(k+1)}$ es continua (pues $\widetilde{x}_{k+1}$ lo es).
		\end{proof}

	\item $x_{n} \tiende{n \to \infty} x$ en $X$.
	
	\begin{proof} Sea $\epsilon > 0$. Queremos probar que existe $N \in \N$ tal que si $n \geq N$, entonces
		$$\|x_{n} - x\|_{X} < \epsilon$$
	
	donde $\|\cdot\|_{X}$ la norma inducida por $d$. Sea $N \in \N$ tal que $n,m \geq N$ implica
		$$\|x_{n} - x_{m}\|_{X} = \sum_{j=1}^{\infty} \sum_{k=0}^{\infty} 2^{-j-k} \frac{\|x_{n}-x_{m}\|_{j,k}}{1 + \|x_{n}-x_{m}\|_{j,k}} < \frac{\epsilon}{2}$$
	
	Por la cota obtenida en (1.1), tomando el límite cuando $m \to \infty$ tenemos
		$$\lim_{m \to \infty} \|x_{n} - x_{m}\|_{X} = \left\|\lim_{m \to \infty} x_{n} - x_{m}\right\|_{X} = \sum_{j=1}^{\infty} \sum_{k=0}^{\infty} 2^{-j-k} \frac{\|x_{n} - x\|_{j,k}}{1 + \|x_{n} - x\|_{j,k}} \leq \frac{\epsilon}{2} < \epsilon$$
	
	mostrando que el límite es en el sentido dado por la norma $\|\cdot\|_{X}$.
	\end{proof}
	\end{afirmacion}

\end{proof}

\newpage
%P2%
\section*{Problema 2}
Sean
	$$C_{b}(\R) := \conj{x \in C(\R)}{\sup_{t \in \R} |x(t)| < \infty}, \qquad
		E := \conj{x \in C_{b}(\R)}{x(k) = 0, \, k \in \Z}$$
		
\begin{enumerate}[(i)]
	%2.a%
	\item Demuestre que $E$ es subespacio cerrado de $C_{b}(\R)$.
	
	\begin{proof}[Solución] Sea $\{x_{n}\}_{n \in \N} \subset E$ sucesión convergente en $C_{b}(\R)$, es decir,
		$$\lim_{n \to \infty} x_{n} = x \in C_{b}(\R)$$
	
	Queremos mostrar que $x \in E$. Sea $k \in \Z$. Por convergencia, dado $\epsilon > 0$ existe $N = N(\epsilon) \in \N$ tal que
		$$\|x - x_{N}\|_{C_{b}} = \sup_{t \in \R} |x(t) - x_{N}(t)| < \epsilon$$
	
	en particular, 
		$$|x(k) - x_{N}(k)| < \epsilon$$ 
	
	y como $x_{N} \in E$, $x_{N}(k) = 0$, luego 
		$$|x(k)| < \epsilon$$
	
	Como este tratamiento vale para todo $\epsilon > 0$ (el valor de $N(\epsilon)$ puede crecer, pero $x_{N(\epsilon)}(k) = 0$ para todo $\epsilon$), debe tenerse que $x(k) = 0$.	\\
	
	Luego, como $x \in C_{b}(\R)$ y $x(k) =0 $ para todo $k \in \Z$ (porque $k$ era arbitrario), entonces $x \in E$ y concluimos que $E$ es cerrado.
	\end{proof}
	
	\newpage
	%2.b%
	\item Determine si $E$ es separable o no. Justifique su respuesta.
	
	\begin{proof}[Solución] No es separable. En efecto, supongamos que existe $D$ denso numerable. Para cada $r \in \R$, con expansión decimal $r = a_{0}.a_{1}a_{2}\ldots$ (si $r \in \Q$ consideramos la expansión decimal periódica) definimos $f_{r}$ por
		$$f_{r}(x) = \twodef{0}{x = k \in \Z}
						{a_{k}}{x = k + \frac{1}{2}, k \in \Z}$$
					
	y unimos estos puntos con rectas. Luego, por construcción $f_{r}$ es continua en $\R$, $f(k) = 0$ para cada $k \in \Z$ y
		$$\sup_{t \in \R} |f_{r}(t)| \leq \max\{|a_{0}|,9\} < \infty$$
	
	por lo que $f_{r} \in E$. Notar además que, si $r_{1} \neq r_{2}$, entonces difieren en algún dígito de su expansión decimal, de manera que
		$$\|f_{r_{1}} - f_{r_{2}}\|_{C_{b}(\R)} \geq 1$$
	
	Entonces, dado $\epsilon \in \left(0,\frac{1}{2}\right)$, sea $x_{r} \in D$ tal que
		$$\|x_{r} - f_{r}\|_{C_{b}(\R)} < \epsilon $$
		
	Si $r_{1} \neq r_{2}$, sucede que $x_{r_{1}} \neq x_{r_{2}}$ pues por lo anterior
		\begin{align*}
			1	&\leq		\|f_{r_{1}} - f_{r_{2}}\|_{C_{b}(\R)}	\\
				&\leq		\|f_{r_{1}} - x_{r_{1}}\|_{C_{b}(\R)} + \|x_{r_{1}} - x_{r_{2}}\|_{C_{b}(\R)} + \|x_{r_{2}} - f_{r_{2}}\|_{C_{b}(\R)}	\\
				&<		2\epsilon + \|x_{r_{1}} - x_{r_{2}}\|_{C_{b}(\R)}
		\end{align*}
	
	De donde
		$$\|x_{r_{1}} - x_{r_{2}}\|_{C_{b}(\R)} > 1- 2\epsilon > 0$$
		
	pues $\epsilon \in \left(0,\frac{1}{2}\right)$. Luego,
		$$\{x_{r}: r \in \R\} \subset D$$
	
	y es un conjunto no numerable (pues las $x_{r}$ son distintas entre ellas y están indexadas por reales), lo que contradice que $D$ sea numerable.
	\end{proof}
\end{enumerate}

\newpage
%P3%
\section*{Problema 3}
Sean
	$$X := C[0,1], \qquad Y := L^{1}(0,1)$$
	$$E_{1} := \clx{L}(X,X), \quad E_{2} := \clx{L}(X,Y), \quad E_{3} := \clx{L}(Y,X), \quad E_{4} := \clx{L}(Y,Y)$$

Sea $K \in C([0,1] \times [0,1])$. Definamos el operador $T_{K}$ por
	$$(T_{K}x)(t) := \int_{0}^{1} K(t,s)x(s) \, ds, \qquad t \in [0,1]$$

\begin{enumerate}[(i)]
	%3.a%
	\item Demuestre que $T_{K} \in E_{j}$, $j = 1,2,3,4$.
	
	\begin{proof}[Solución] Observemos que $K$ es continua en un compacto y por lo tanto es uniformemente continua. Además, la linealidad de $T_{K}$ sigue directamente de la linealidad de la integral. Observar también que $X \subset Y$ pues cada función continua en $[0,1]$ es acotada y como $[0,1]$ también lo es, la integral (del valor absoluto) es finita. Sea
		$$M := \max_{(u,v) \in [0,1]^{2}} |K(u,v)|$$
	
	el cual existe porque $|K|$ es una función continua en un conjunto compacto.
	
	\textbf{Afirmación 1} Si $x \in X \supset Y$, entonces $T_{K}x \in X \subset Y$.
	
	\begin{proof} Sea $x \in Y$. Dados $t_{0}, t_{1} \in [0,1]$ distintos,
		\begin{align*}
			\left| (T_{K}x)(t_{1}) - (T_{K}x)(t_{0})\right|
				=	\left| \int_{0}^{1} (K(t_{1}, s) - K(t_{0},s))x(s) \, ds \right|
				\leq	\int_{0}^{1} |K(t_{1},s) K(t_{0},s)| |x(s)| \, ds
		\end{align*}
		
	Sea $\epsilon > 0$, como $x \in Y$, entonces $B := \int_{0}^{1} |x(s)| \, ds < \infty$ y por la continuidad uniforme de $K$ existe $\delta > 0$ tal que $\|(t_{1}, s) - (t_{0},s)\| = |t_{1} - t_{0}| < \delta$ implica
		$$|K(t_{1}, s) - K(t_{0}, s)| < \epsilon B^{-1}	\qquad \paratodo s \in [0,1]$$
		
	Luego, si $|t_{0} - t_{1}| < \delta$, entonces
		$$\left| (T_{K}x)(t_{1}) - (T_{K}x)(t_{0})\right| < \epsilon B^{-1} \int_{0}^{1} |x(s)| \, ds = \epsilon$$		
	\end{proof}

	\textbf{Afirmación 2} $T_{K} \in E_{1}$.
		
	\begin{proof} Por la afirmación 1, $T_{K}x :X \to X$ y ya sabemos que es lineal. Falta ver que es acotada. Dado $x \in X$, con $\|x\|_{X} = \max_{t \in [0,1]} |x(t)| = 1$, tenemos que
		\[ \begin{aligned}
			\|T_{K}x\|_{X}
				&=	\max_{t \in [0,1]} \left| \int_{0}^{1} K(t,s)x(s) \, ds \right|	\\
				&\leq	\max_{t \in [0,1]} \int_{0}^{1} |K(t,s)||x(s)| \, ds	\\
				&\leq	\max_{t \in [0,1]} \int_{0}^{1} |K(t,s)| \, ds	&&\big/ \, \|x\|_{X} = 1	\\
				&\leq	\max_{t \in [0,1]} M = M
		\end{aligned} \tag{3.1}\]
		
	por lo que $T_{K}$ es acotada y $\|T_{K}\|_{E_{1}} \leq M$.
	\end{proof}

	\textbf{Afirmación 3} $T_{K} \in E_{2}$.
		
	\begin{proof} Nuevamente, por la afirmación 1, $T_{K}x : X \to Y$ y ya sabemos que es lineal. Ahora, dado $x \in X$, con $\|x\|_{X} = 1$ tenemos que
		\[ \begin{aligned}
			\|T_{K}x\|_{Y}
				&=	\int_{0}^{1} \left| \int_{0}^{1} K(t,s)x(s) \, ds \right| \, dt	\\
				&\leq	\int_{0}^{1} \int_{0}^{1} |K(t,s)| |x(s)| \, ds \, dt	\\
				&\leq	\int_{0}^{1} \int_{0}^{1} |K(t,s)| \, ds \, dt	&&\big/ \, \|x\|_{X} = 1	\\
				&\leq	M
		\end{aligned} \tag{3.2} \]
	
	por lo que $\|T_{K}\|_{E_{2}} \leq M$ y $T_{K}$ es acotada.
	\end{proof}

	\newpage
	\textbf{Afirmación 4} $T_{K} \in E_{3}$.
		
	\begin{proof} Por la afirmación 1, $T_{K}: Y \to X$ y es lineal. Luego, dado $x \in Y$, con 
		$$\|x\|_{Y} = \int_{0}^{1} |x(s)| \, ds = 1$$
		
	tenemos que
		\[ \begin{aligned}
			\|T_{K}x\|_{X}
				&=	\max_{t \in [0,1]} \left| \int_{0}^{1} K(t,s)x(s) \, ds \right|	\\
				&\leq	\max_{t \in [0,1]} \int_{0}^{1} |K(t,s)||x(s)| \, ds	\\
				&\leq	\max_{t \in [0,1]} M	&&\big/ \, \|x\|_{Y} = 1	\\
				&=	M
		\end{aligned} \tag{3.3} \]
		
	Luego $T_{K}$ es acotada y $\|T_{K}\|_{E_{3}} \leq M$.
	\end{proof}
	
	\textbf{Afirmación 5} $T_{K} \in E_{4}$.
	
	\begin{proof} Ya hemos visto (afirmación 1) que $T_{K}: Y \to Y$ y que es lineal. Luego, dado $x \in Y$ con $\|x\|_{Y} = 1$,
		\[ \begin{aligned}
			\|T_{K}x\|_{Y}
				&=	\int_{0}^{1} \left| \int_{0}^{1} K(t,s)x(s) \, ds \right| \, dt	\\
				&\leq	\int_{0}^{1} \int_{0}^{1} |K(t,s)| |x(s)| \, ds \, dt	\\
				&\leq	\int_{0}^{1} M \, dt										&&\big/\,  \|x\|_{Y} = 1	\\
				&=	M
		\end{aligned} \tag{3.4}\]
	
	Luego $T_{K}$ es acotada y $\|T_{K}\|_{E_{4}} \leq M$
	\end{proof}
	\end{proof}

	\newpage
	%3.b%
	\item Si $K = \overline{K} > 0$, encuentre las normas $\|T_{K}\|_{E_{j}}$, $j = 1,2,3,4$.
	
	\begin{proof}[Solución] Como $K > 0$, sean
		\begin{align*}
			L_{1}	&:=	\max_{t \in [0,1]} \int_{0}^{1} K(t,s) \, ds > 0	\\
			L_{2}	&:=	\int_{[0,1]^{2}} K(t,s) \, ds dt > 0	
		\end{align*}
		
	Observar que $L_{1} < \infty$ pues $L_{1} = \|T_{K} \cdot \mathbf{1}\|_{X}$ con $\mathbf{1}$ la función constante igual a 1 y $L_{2}$ existe porque $K$ es continua. Notar además que $\|\mathbf{1}\|_{X} = 1$. Por (3.1), para todo $x \in X$ con $\|x\|_{X} = 1$, como $K > 0$
		$$\|T_{K}x\|_{X} \leq \max_{t \in [0,1]} \int_{0}^{1} K(t,s) \, ds = L_{1}$$
		
	De manera que $\|T_{K}\|_{E_{1}} \leq L_{1}$ y como $L_{1}$ se alcanza para $x = \mathbf{1}$, entonces $\|T_{K}\|_{E_{1}} = L_{1}$. Ahora, por (3.2), para todo $x \in X$ con $\|x\|_{X} = 1$,
		$$\|T_{K}x\|_{Y} \leq \int_{0}^{1} \int_{0}^{1} K(t,s) \, ds \, dt = L_{2}$$
	
	De manera que $\|T_{K}\|_{E_{2}} \leq L_{2}$ y como $L_{2}$ se alcanza para $x= \mathbf{1}$, entonces $\|T_{K}\|_{E_{2}} = L_{2}$.	\\
	
	Para las demás normas, sea $(\mathbf{u}, \mathbf{v}) \in [0,1]^{2}$ el punto para el cual
		$$K(\mathbf{u}, \mathbf{v}) = |K(\mathbf{u}, \mathbf{v})| = M$$
	
	Como $K$ es uniformemente continua, entonces existe
		$$L_{3} := \max_{s \in [0,1]} \int_{0}^{1} K(t,s) \, dt = \int_{0}^{1} K(t,\mathbf{s}) \, dt$$
	
	Llamemos $\clx{I}_{n} := \left[\mathbf{v} - \frac{1}{2n}, \mathbf{v} + \frac{1}{2n}\right]$ (de largo $n^{-1}$). Consideremos la sucesión $\{x_{n}\} \subset Y$ con $x_{n}(t) = n\uno_{\clx{I}_{n}}(t)$. Tenemos que
		$$\|x_{n}\|_{Y} = \int_{0}^{1} n \uno_{\clx{I}_{n}}(s) \,ds = n \cdot \frac{1}{n} = 1$$
	
	Y además, como $K$ es continua
		$$(T_{K}x_{n})(t) = \int_{0}^{1} K(t,s)x_{n}(s) \, ds = \frac{1}{1/n} \int_{\clx{I}_{n}} K(t,s) \, ds \tiende{n \to \infty} K(t,\mathbf{v})$$
	
	De esta forma, usando que $T_{K}x_{n}$ es continua (y positiva) para todo $n$,
		\begin{align*}
			\lim_{n \to \infty} \|T_{K}x_{n}\|_{X}
				&=	\lim_{n \to \infty} \max_{t \in [0,1]} \left| (T_{K}x_{n})(t) \right|	\\
				&=	\max_{t \in [0,1]} \lim_{n \to \infty} (T_{K}x_{n})(t)	\\
				&=	\max_{t \in [0,1]} K(t,\mathbf{v})	\\
				&=	K(\mathbf{u}, \mathbf{v})	\\
				&=	M
		\end{align*}
	
	Como por (3.3) $\|T_{K}\|_{E_{3}} \leq M$, entonces $\|T_{K}\|_{E_{3}} = M$. Finalmente, siguiendo el raciocinio de (3.4), para $x \in Y$ con $\|x\|_{Y} = 1$ tenemos (usando Fubini, ya que la integral doble es de términos positivos y convergente)
		\[ \begin{aligned}
			\|T_{K}x\|_{Y}
				&\leq	\int_{0}^{1} \int_{0}^{1} K(t,s) |x(s)| \, ds \, dt	\\
				&=	\int_{0}^{1} |x(s)| \int_{0}^{1} K(t,s) \, dt \, ds	\\
				&\leq	L_{3} \int_{0}^{1} |x(s)| \, ds	\\
				&=	L_{3}									&&\big/\, \|x\|_{Y} = 1
		\end{aligned}\]
	
	lo que dice que $\|T_{K}\|_{E_{4}} \leq L_{3}$, pero usando una sucesión similar a la anterior, con $\clx{J}_{n} = \left[\mathbf{s} - \frac{1}{2n}, \mathbf{s} + \frac{1}{2n}\right]$ y $x_{n}(t) = n\uno_{\clx{J}_{n}}(t)$ (de manera que $\|x_{n}\|_{Y} = 1$) obtenemos
		\begin{align*}
			\lim_{n \to \infty} \|T_{K}x_{n}\|_{Y}
				&=	\lim_{n \to \infty} \int_{0}^{1} \int_{0}^{1} K(t,s)x_{n}(s) \, ds \, dt	\\
				&=	\int_{0}^{1}  \lim_{n \to \infty} \int_{0}^{1} K(t,s)x_{n}(s) \, ds \, dt	\\
				&=	\int_{0}^{1} K(t,\mathbf{s}) \, dt	\\
				&=	L_{3}
		\end{align*}
	
	donde la permutación del límite con la integral se sustenta en que
		$$\int_{0}^{1} K(t,s)x_{n}(s) \, ds = \frac{1}{1/n}\int_{\clx{I}_{n}} K(t,s) \, ds \leq \frac{1}{1/n} \cdot M \cdot \frac{1}{n} = M$$
	
	y $M$ es trivialmente integrable en $[0,1]$. Con esto concluimos que $\|T_{K}x\|_{E_{4}} = L_{3}$.
	\end{proof}

\end{enumerate}

\newpage
%P4%
\section*{Problema 4}
Sean
	\begin{align*}
		G_{p}
			&:=	\conj{x \in \ell^{p}(\N)}{x_{j} = 0, \, j \geq 4}, \qquad p \in [1,\infty]	\\
		\lambda(x)
			&:=	x_{1} + x_{2} - 2x_{3}, \qquad x \in G_{p}
	\end{align*}

\begin{enumerate}[(i)]
	%4.a%
	\item Demuestre que $\lambda \in G_{p}^{*}$ y encuentre su norma.
	
	\begin{proof}[Solución] Es claro que $\lambda: G_{p} \to \C$ para cada $p \in [1,\infty]$ y que es lineal (es una combinación lineal de coordenadas). Como además, para $\|x\|_{\ell^{p}(\N)} = 1$ con $x \in G_{p}$, si $p' \in [1,\infty]$ es tal que $\frac{1}{p'} + \frac{1}{p} = 1$ (con la convención de que $p' = \infty$ si $p = 1$ y viceversa), entonces
		\[ \begin{aligned}
			|\lambda(x)|
				&\leq		|x_{1}| + |x_{2}| + 2|x_{3}|	\\
				&=		\| f_{0} x\|_{1}						&&\big/\, f_{0} = (1,1,-2,0,\ldots)	\\
				&\leq		\|f_{0}\|_{\ell^{p'}(\N)} \|x\|_{\ell^{p}(\N)}	&&\big/\, \text{H\"older}	\\
				&=		\|f_{0}\|_{\ell^{p'}(\N)}
		\end{aligned} \tag{4.1} \]
	
	y como $f_{0} \in f \subset \ell^{p'}$ para todo $p' \in [1,\infty]$ (con $f$ las sucesiones ``finitas''), entonces $\|f_{0}\|_{\ell^{p'}(\N)}$ es finito y $\lambda$ es acotado. Con esto concluimos que $\lambda \in G_{p}^{*}$ y más aún $\|\lambda\|_{G_{p}^{*}} \leq \|f_{0}\|_{\ell^{p'}(\N)}$.	\\
	
	Para encontrar su norma, comencemos por el caso $p = 1$, en cuyo caso $p' = \infty$ y tenemos $\|f_{0}\|_{\ell^{\infty}(\N)} = 2$. Notemos que $e_{3} = (0,0,1,0,\ldots) \in G_{1}$ y $\|e_{3}\|_{\ell^{1}(\N)} = 1$. Además,
		$$|\lambda(e_{3})| = |-2| = 2 = \|f_{0}\|_{\ell^{\infty}(\N)}$$
	
	por lo que $\|\lambda\|_{G_{1}^{*}} = 2$. Ahora, supongamos $p = \infty$, en cuyo caso $p' = 1$ y
		$$\|f_{0}\|_{\ell^{1}(\N)} = 1 + 1 + 2 = 4$$
	
	Sea $x_{0} = (1,1,-1,0,\ldots) \in G_{\infty}$ y con $\|x_{0}\|_{\ell^{\infty}(\N)} = 1$. Tenemos que
		$$|\lambda(x_{0})| = |4| = 4$$

	por lo que $\|\lambda\|_{G_{\infty}^{*}} = 4$.	\\
	
	Ahora, sean $p \in (1,\infty)$ y $x \in G_{p}$ con $\|x\|_{\ell^{p}(\N)} = 1$. Para tener igualdad entre $\|\lambda\|_{G_{p}^{*}}$ y $\|f_{0}\|_{\ell^{p'}(\N)}$ debe darse la igualdad en H\"older, lo que ocurre si y solo si existe $\beta \geq 0$ tal que
		\[ |x_{j}|^{p} = \beta |(f_{0})_{j}|^{p'} \qquad j = 1,2,3	\tag{4,2} \]
	
	Sumando sobre $j$ tenemos que
		$$1 = \beta \sum_{j=1}^{3} |(f_{0})_{j}|^{p'} = \beta \|f_{0}\|_{\ell^{p'}(\N)}^{p'}$$
	
	de donde 
		$$\beta = \| f_{0}\|_{\ell^{p'}(\N)}^{-p'} = \left( 1 + 1 + 2^{p'} \right)^{-1} = \left(2 + 2^{p'}\right)^{-1}$$.
	
	Luego, tomando
		$$x_{0} = \beta^{1/p} ( 1, 1, -2^{p'/p}, 0, \ldots ) = \beta^{1/p} \Big( (f_{0})_{1}^{p'/p} , (f_{0})_{2}^{p'/p}, -(-f_{0})_{3}^{p'/p} \Big) $$
	
	se tiene por construcción que $\|x_{0}\|_{\ell^{p}(\N)} = 1$ (pues se verifica (4.2)), $x_{0} \in G_{p}$ y usando que $\frac{1}{p} + \frac{1}{p'} = 1$ implica $p' + p = pp'$,
		\begin{align*}
			\lambda(x_{0})
				&=	\beta^{1/p} \left( 2 + 2^{\frac{p'}{p} + 1} \right)	\\
				&=	\beta^{1/p} \left(2 + 2^{\frac{p' +p}{p}} \right)	\\
				&=	\beta^{1/p} \left(2 + 2^{\frac{p'p}{p}} \right)	\\
				&=	\beta^{1/p}\beta^{-1}	\\
				&=	\beta^{-1/p'}	\\
				&=	\|f_{0}\|_{\ell^{p'}(\N)}
		\end{align*}
	
	por lo que $\|\lambda\|_{G_{p}^{*}} = \|f_{0}\|_{\ell^{p'}(\N)}$.
	\end{proof}
	
	\newpage
	%4.b%
	\item Demuestre que si $p \in (1,\infty)$, entonces existe único $\Lambda \in \ell^{p}(\N)^{*}$ tal que $\Lambda \big|_{G_{p}}= \lambda$ y $\|\Lambda\|_{\ell^{p}(\N)^{*}} = \|\lambda\|_{G_{p}^{*}}$. Encuentre $\Lambda$ explícitamente.
	
	\begin{proof}[Solución] Sea
		$$\Lambda(x) = x_{1} + x_{2} - 2x_{3}, \qquad x \in \ell^{p}(\N)$$
	
	La linealidad de $\Lambda$ sigue de la parte (i) y además es claro que $\Lambda\big|_{G_{p}} = \lambda$. Para ver que es acotado, si $f_{0} \in \ell^{p'}(\N)$ es como arriba, entonces por (4.1), si $\|x\|_{\ell^{p}(\N)} = 1$,
		$$|\Lambda(x)| \leq \|f_{0}\|_{\ell^{p'}(\N)} = \|\lambda\|_{G_{p}^{*}}$$
		
	luego $\Lambda \in \ell^{p}(\N)^{*}$. Pero más aún tenemos que
		$$\Lambda(x) = \sum_{j=1}^{\infty} (f_{0})_{j}x_{j}, \qquad x = (x_{1}, \ldots) \in \ell^{p}(\N)$$
	
	y, por el teorema de Riesz,
		$$\|\Lambda\|_{\ell^{p}(\N)^{*}} = \|f_{0}\|_{\ell^{p'}(\N)} = \|\lambda\|_{G_{p}^{*}}$$
	
	Para la unicidad, sea $\varphi \in \ell^{p}(\N)^{*}$ otro funcional que satisface las mismas propiedades. Luego,
		\[ \|\varphi\|_{\ell^{p}(\N)^{*}} = \|\lambda\|_{G_{p}^{*}} = \|f_{0}\|_{\ell^{p'}(\N)}	\tag{4.3} \]
	
	Sea $y \in \ell^{p'}(\N)$ dada por el teorema de Riesz aplicado para $\varphi$. Sean $e_{j} = (\delta_{ij})_{i \in \N}$ para $j = 1, 2,3$. Notar que $e_{j} \in G_{p}$ y
		\begin{align*}
			(f_{0})_{j}	&=	\lambda(e_{j}) = \varphi(e_{j}) = \sum_{k=1}^{\infty} (e_{j})_{k}y_{k} = y_{j}
		\end{align*}
	
	Luego las primeras 3 coordenadas de $y$ coinciden con las de $f_{0}$. Supongamos que $y \neq f_{0}$, eso implica que existe $j \geq 4$ tal que $y_{j} \neq 0$ y por lo tanto
		$$\|y\|_{\ell^{p'}(\N)}^{p} = \sum_{j=1}^{\infty} |y_{j}|^{p'} > \sum_{j=1}^{\infty} |(f_{0})_{j}|^{p'} = \|f_{0}\|_{\ell^{p'}(\N)}$$
	
	lo que en particular implica que $\|\varphi\|_{\ell^{p}(\N)^{*}} > \|f_{0}\|_{\ell^{p'}(\N)}$ contradiciendo (4.3). Así, $y = f_{0}$ y por lo tanto
		$$\varphi(x) = \sum_{j=0}^{\infty} (f_{0})_{j}x_{j} = x_{1} + x_{2} - 2x_{3} = \Lambda(x) \qquad \paratodo x \in \ell^{p}(\N)$$
	
	concluyendo que $\Lambda = \varphi$.
	\end{proof}
	
	%4.c%
	\item Demuestre que si $p = 1$, entonces existe número infinito de funcionales $\Lambda \in \ell^{1}(\N)^{*}$ tales que $\Lambda_{G_{1}} = \lambda$ y $\|\Lambda\|_{\ell^{1}(\N)^{*}} = \|\lambda\|_{G_{1}^{*}}$. Encuentre explícitamente el conjunto de tales funcionales $\Lambda$.
	
	\begin{proof}[Solución] Sea $M = \conj{x \in \ell^{\infty}(\N)}{x_{1} = 1, x_{2} = 1, x_{3} = -2, \|x\|_{\ell^{\infty}(\N)} = 2}$, es decir, $M$ contiene todas las sucesiones que coinciden con $f_{0}$ en las primeras 3 coordenadas. La última condición asegura que $|x_{j}|$ no es mayor a 2 para $j \geq 4$. Observar además que $M$ es un conjunto infinito. Para $y \in M$, sea
		$$\Lambda_{y}(x) = \sum_{j=1}^{\infty} x_{j}y_{j} \qquad x \in \ell^{1}(\N)$$
	
	Observar que cada $\Lambda_{y}$ es distinto y la definición es correcta en virtud de que para cada $x \in \ell^{1}(\N)^{*}$,
		\[ |\Lambda_{y}(x)| \leq \sum_{j=1} |x_{j}||y_{j}| \leq \|x\|_{\ell^{1}(\N)} \|y\|_{\ell^{\infty}(\N)} = 2\|x\|_{\ell^{1}(\N)} < \infty	 \]
	
	La linealidad de $\Lambda_{y}$ para cada $y \in M$ es clara de la linealidad de la suma y del límite. Como además, por lo anterior, si $\|x\|_{\ell^{1}(\N)} = 1$, entonces $|\Lambda_{y}(x)| \leq 2$ concluyendo, no solo que $\Lambda_{y} \in \ell^{1}(\N)^{*}$ para cada $y \in M$, sino que también $\|\Lambda_{y}\|_{\ell^{1}(\N)^{*}} \leq 2$.  Notar además que si $x \in G_{p}$, $x_{j} = 0$ para $j \geq 4$ y
		$$\Lambda_{y}(x) = \sum_{j=1}^{3} x_{j}y_{j} = x_{1} + x_{2} - 2x_{3} = \lambda(x) \qquad \paratodo y \in M$$
	
	y $\Lambda_{y}\big|_{G_{p}} = \lambda$. Falta probar la igualdad de las normas, pero esto sigue directamente de que para $e_{3} = (0,0,1,0,\ldots)$, $\|e_{3}\|_{\ell^{1}(\N)} = 1$ y $|\Lambda_{y}(e_{3})| = |y_{3}| = 2$ por lo que la norma se alcanza y es igual a 2. Concluimos que el conjunto $\clx{F} = \{\Lambda_{y} : y \in M\}$ verifica lo pedido y es infinito porque $M$ lo es y cada $\Lambda_{y}$ es distinta. Falta verificar que no hay otros funcionales con las propiedades dadas. Sea $\varphi \in \ell^{1}(\N)^{*}$ con estas características y, por el teorema de Riesz, sea $y \in \ell^{\infty}(\N)$ tal que
		$$\varphi(x) = \sum_{j=1}^{\infty} x_{j}y_{j} \qquad x \in \ell^{1}(\N)$$
	
	como $\varphi\big|_{G_{1}} = \lambda$, entonces para $e_{j} = (\delta_{ij})_{i \in \N}$, j = 1,2,3, tenemos que
		$$y_{j} = \varphi(e_{j}) = \lambda(e_{j}) = (f_{0})_{j}, \qquad j = 1,2,3$$
	
	luego $y_{0} = y_{1} = 1$ e $y_{2} = -2$. Además, por el teorema de Riesz, $\|\varphi\|_{\ell^{1}(\N)^{*}} = \|y\|_{\ell^{\infty}(\N)}$ y
		\[ \|y\|_{\ell^{\infty}(\N)}
				=	\|\varphi\|_{\ell^{1}(\N)^{*}}
				=	\|\lambda\|_{G_{1}}
				=	\|f_{0}\|_{\ell^{\infty}(\N)}
				=	2 \]
	
	por lo que $y \in M$ y por lo tanto $\varphi = \Lambda_{y}$.
	\end{proof}
\end{enumerate}
\end{document}