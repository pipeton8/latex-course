%\documentclass[11pt]{beamer}
\documentclass[11pt, handout]{beamer}

%: Paquetes
\usepackage{fontenc}
\usepackage[spanish]{babel}
\usepackage[utf8]{inputenc}

%: Hyperrref setup
\hypersetup{colorlinks=true,
			allcolors=black,
			urlcolor=blue}

%: Tema
\usetheme{CambridgeUS}	
\usecolortheme{beaver}
\usefonttheme{professionalfonts}

%: Beamer customization
\beamertemplatenavigationsymbolsempty
\setbeamercovered{transparent} 

%: Argumentos para la portada
\title{Introducción al taller de \LaTeX}
\author{Felipe del Canto}
\date{Marzo, 2021}

%: Documento
\begin{document}

\frame{\titlepage}

\begin{frame}{?`Qué es \LaTeX?}
	
	\begin{itemize}
		\item {\LaTeX} es un compositor de texto, tal como Word.\pause
		\vfill
		
		\item Sin embargo, en Word ``lo que escribes es lo que ves''.\pause
		\vfill
		
		\item En \LaTeX, debemos escribir ``código'' y luego compilar.\smallskip
			\begin{itemize}
				\item Es decir, lo que escribimos no es igual al producto final.\footnote{Existe un programa llamado LyX en el cual se escribe lo que se ve, pero no lo recomiendo, sobre todo si desean compartir el documento con otros.}
			\end{itemize}

	\end{itemize}
\end{frame}

\begin{frame}{Entonces, ?`por qué \LaTeX?}
	
	\begin{itemize}
		\item {\LaTeX} tiene muchas ventajas sobre Word para escribir textos formales.\smallskip
			\begin{itemize}
				\item Sobre todo artículos científicos.\pause
			\end{itemize}
		\vfill
		
		\item Algunas ventajas:\smallskip
			\begin{itemize}
				\item Formato automatizado (!`adiós fotos que se pierden!).\smallskip
				\item Amplia personalización.\smallskip
				\item Soporte para distintos idiomas.\smallskip
				\item Facilidad para escribir texto matemático.\smallskip
				\item Incorporación ``simple'' de bibliografía.
			\end{itemize}
	\end{itemize}
\end{frame}

\begin{frame}{?`Cómo y dónde puedo usar \LaTeX?}
	
	\begin{itemize}
		\item !`Se puede hacer en Windows, Mac y Linux!\pause
		\vfill
		
		\item Para hacerlo se necesitan dos ingredientes:\smallskip
			\begin{itemize}
				\item Una ``distribución'' (básicamente hay que instalar {\LaTeX}).\smallskip
				\item Un editor, que es donde se escribe el ``código'' y se compila.\pause
			\end{itemize}
		
		\vfill
		
		\item Pueden obtener la ``distribución'' para su sistema operativo \href{https://www.latex-project.org/get}{aquí}.\smallskip
			\begin{itemize}
				\item Usuarios de Mac obtienen un editor automáticamente.\smallskip
				\item Para Windows o Linux se recomiendan \href{https://www.xm1math.net/texmaker/download.html}{Texmaker} o \href{https://www.texstudio.org}{TeXstudio}.
			\end{itemize}

	\end{itemize}
\end{frame}

\begin{frame}{?`Y si no quiero instalar cosas?}
	
	\begin{itemize}
		\item Hay alternativas para trabajar online en \LaTeX.\pause
		\vfill
		
		\item La más famosa es Overleaf (\href{http://overleaf.com}{link}).\smallskip
			\begin{itemize}
				\item Se puede colaborar con otras personas (máximo 1 gratis).\smallskip
				\item Se puede sincronizar con Dropbox (solo la versión pagada).\smallskip
				\item Se puede sincronizar con Github.\smallskip
				\item Revisa automáticamente errores de código.\pause
			\end{itemize}
		\vfill
			
		\item Existe también Papeeria (\href{https://papeeria.com}{link}).
			\begin{itemize}
				\item Es muy similar a Overleaf pero sin pagar por las funcionalidades.
			\end{itemize}
	\end{itemize}
\end{frame}

\begin{frame}{Pero no siempre llevo el computador a la U \ldots}
	
	\begin{itemize}
		\item Por si fuera poco !`podemos usar {\LaTeX} en iOS o Android!\pause
		\vfill
		
		\item Alternativas en iOS (en orden de mi preferencia):\smallskip
			\begin{enumerate}
				\item LaTeX Editor Tex Pro (\href{https://apps.apple.com/cl/app/latex-editor-tex-pro/id1486802741?l=en}{link}).\smallskip
				\item TeX Writer (\href{https://apps.apple.com/cl/app/tex-writer-latex-on-the-go/id552717222?l=en}{link}).\smallskip
				\item Texpad (\href{https://apps.apple.com/cl/app/texpad-latex-editor/id550419159?l=en}{link}).\smallskip
				\item VerbTeX (\href{https://apps.apple.com/cl/app/verbtex-latex-editor/id560869163?l=en}{link}).\pause
			\end{enumerate}
		\vfill
			
		\item Alternativas en Android (sin order particular):\smallskip
			\begin{enumerate}
				\item VerbTeX (\href{https://play.google.com/store/apps/details?id=verbosus.verbtexpro}{link}).\smallskip
				\item TeX Writer (\href{https://play.google.com/store/apps/details?id=com.litchie.texwriter}{link}).\pause
			\end{enumerate}
		\vfill
		
		\item También se puede usar Overleaf y Papeeria (menos recomendable). 
		
	\end{itemize}
\end{frame}

\begin{frame}{Antes de empezar\ldots}
	\begin{itemize}
		\item Este curso está pensado para ser totalmente trabajado desde Overleaf.
		\vfill
		
		\item Para eso, sigan los siguientes pasos:\smallskip
			\begin{enumerate}
				\item Crear un usuario en Overleaf (pueden ingresar con su correo UC).\smallskip
				
				\item Acceder al proyecto a través de \href{https://www.overleaf.com/read/fxyhkqyndsrx}{este link}.\smallskip
				
				\item Una vez adentro, presionar \texttt{Menú} y luego \texttt{Copiar proyecto}.
				
				
			\end{enumerate}
		\vfill
		
		\item Si no desean usar Overleaf, pueden acceder al curso completo \href{https://github.com/pipeton8/latex-course}{aquí}. 
	\end{itemize}

\end{frame}


\end{document}

