% Hoy crearemos nuestra primera presentación en Beamer
\documentclass[11pt]{beamer}	% Aquí llamamos la clase beamer, podríamos crear nuestra propia clase, con opciones y otras cosas para hacerlo más inteligente

% Paquetes %
% Los paquetes se llaman como a cualquier documento LaTeX, porque básicamente estamos escribiendo un texto de LaTeX
\usepackage{tutorialLaTeX}	% Incluso, como es de esperar, podemos incorporar nuestro propio paquete (por esto tiene sentido no incorporar las configuraciones propias de una clase dentro de un paquete)

% Para beamer, tenemos que seleccionar un cierto ``tema'', que corresponde a la apariencia de la presentación. Opcionalmente podemos determinar el tipo de letra, el color y la forma en que enumerate e itemize se ven
% Pueden revisar los colores y temas en forma matricial en el siguiente link:
% https://hartwork.org/beamer-theme-matrix/

% Los diferentes tipos de letra (los temas y colores también, pero más desordenados) están en
% http://deic.uab.es/~iblanes/beamer_gallery/individual/Boadilla-default-default.html

% Para definir el tema y el color y el tipo de letra, usamos la siguiente sintaxis
\usetheme{Madrid}	
\usecolortheme{seahorse}
\usefonttheme{professionalfonts}	% Con este tipo de letra quedan mejor las fórmulas de LaTeX 

% Ahora, incluimos el título del texto y nuestro nombre
\title{Primera presentación usando Beamer}
\author{Taller de \LaTeX}
% Alternativamente también podemos poner la fecha con \date

% Documento %
\begin{document}
% Comenzamos con la portada de nuestra presentación.
% Portada %
\frame{\titlepage}

% La unidad básica de beamer es el ``frame''.
% Cada nuevo frame se llama usando la siguiente sintaxis:
%	\begin{frame}[Opciones]
%		\frametitle{Título del frame}	
%		Material del frame
%	\end{frame}

% Las opciones del frame pueden ser:
%	- ``t, c, b'' para alineación vertical del contenido (top, center, bottom)
%	- ``plain'', que elimina el encabezado, el pie y las barras laterales (si hay). Este último es especialmente útil si queremos incluir una imagen de tamaño grande, que se vería extraña si se enmarca con esos elementos.
%	- ``squeeze'', que comprime todo lo posible los espacios verticales.

% Vale decir que no siempre un ``frame'' es una página del PDF que sale, pero lo veremos más adelante.

% Nuestro primer frame será el índice:
%	Índice	%
\begin{frame}
	\frametitle{Índice}
	\tableofcontents
\end{frame}

% Uso de frames %
% Las secciones deben ir al comienzo del primer frame que corresponde a esa sección. Recordar que, al igual que en un texto, nosotros debemos preocuparnos de la organización temática, es decir, de las secciones y subsecciones.
\section{Opciones de frames}
\begin{frame}
	\frametitle{\secname}
	
	El frame por defecto tiene el texto alineado verticalmente al centro. Es equivalente a usar la opción \texttt{[c]} al comenzar el frame.
\end{frame}

\begin{frame}[t]
	\frametitle{\secname}
	
	Pero podríamos alinearlo en la parte superior, usando la opción \texttt{[t]}.
	
\end{frame}

\begin{frame}[b]
	\frametitle{\secname}
	
	O en la parte inferior, usando la opción \texttt{[b]}.
	
\end{frame}

\begin{frame}
	\frametitle{\secname}	% Podemos 
	
	Si además queremos utilizar una imagen muy grande, conviene usar la opción \texttt{plain}, para que beamer se deshaga de todas las barras laterales, superiores e inferiores.
\end{frame}

\begin{frame}[plain]
	\begin{center}
		\includegraphics[scale=0.35]{pobrepatito.jpg}
	\end{center}
\end{frame}

% Entorno Block %
\section{Entorno Block}
% A continuación veremos uno de los entornos más usados en Beamer, el entorno Block que, como su nombre lo dice,
% encierra en ``bloques'' el texto
\begin{frame}
	\frametitle{\secname}
	
	% Comenzamos llamando al entorno
	\begin{block}{Primer bloque con título}	% Con esto le ponemos título
	% Dentro de cualquier frame, podemos incorporar nuestros conocidos entornos enumerate e itemize:
	\begin{itemize}
		\item Podemos escribir esto.
		\item Esto
		\item Y esto 
	\end{itemize}
	
	E incluso podemos escribir \alert{fuera del entorno itemize}.	% \alert es un comando especial de beamer, que escribe en rojo para enfatizar 
	\end{block}
	
	Pero también podemos poner bloques sin título:
	
	\begin{block}{}
		Como este bloque
	\end{block}
\end{frame}

\begin{frame}
	\frametitle{\secname}
	
	Podemos también incorporar bloques con significado, que cambian de color. Para ello tenemos dos entornos, el entorno \texttt{alertblock}:
		\begin{alertblock}{Puede llevar título}
			y el color del bloque es rojo
		\end{alertblock}
	
	Y también tenemos el entorno \texttt{exampleblock}:
		\begin{exampleblock}{Que también puede llevar título}
			y es de color verde
		\end{exampleblock}
\end{frame}

% Animaciones %
\section{Animaciones}
% en esta sección trataremos los casos donde un frame no es necesariamente una diapositiva.
	%%	El comando \textttt{pause} %%
\subsection{El comando \texttt{pause}}
\begin{frame}
	\frametitle{\secname : \subsecname}
	
	% El comando más simple para animar es ``pause'', el cual genera una diapositiva adicional por cada vez que aparece y que corta la diapositiva justo en donde se le llama
	\begin{block}{Mostrando el comando \texttt{pause}}
	\begin{enumerate}
		\item En la primera diapositiva solo se ve este elemento.	\pause
		\item Luego se ve este.							\pause
		\item Y finalmente se ve este.						
	\end{enumerate}
	\end{block}
\end{frame}

\begin{frame}
	\frametitle{\secname: \subsecname}
		
	Pero también es posible que los demás elementos \alert{sí} se vean, usando el comando \texttt{\textbackslash setbeamercovered\{Opción\}}. Este puede llamarse en el preámbulo (si queremos un comportamiento general) o en algún frame en particular (si queremos un comportamiento local). Las opciones son:
		\begin{itemize}
			\item \texttt{invisible}: Se encuentra por defecto y deja los demás elementos invisibles.
			\item \texttt{trasnparent}: El texto se ve casi transparente. Podemos variar la opacidad con el comando \texttt{transparent = X}, con $0 \leq \texttt{X} \leq 100$.
			\item \texttt{dynamic}: A medida que pasa el tiempo, el texto no descubierto varía su grado de visibilidad.		
		\end{itemize}
\end{frame}

\begin{frame}
	\setbeamercovered{transparent}
	\frametitle{\secname : \subsecname}
	
	Por ejemplo, en el caso anterior, si usamos \texttt{transparent} tenemos:
		
	\begin{block}{Mostrando el comando \texttt{\textbackslash pause}}
	\begin{enumerate}
		\item En la primera diapositiva solo se ve este elemento.	\pause
		\item Luego se ve este.							\pause
		\item Y finalmente se ve este.						
	\end{enumerate}
	\end{block}
\end{frame}
	
	%% Rangos para animaciones %% 
\subsection{Rangos para animaciones}
\begin{frame}
	\frametitle{\secname : \subsecname}
	
	Para los comandos que veremos a continuación, es posible asignar rangos, para que los elementos se muestren en el orden que nosotros queremos, eliminando la limitación del comando \texttt{\textbackslash pause}.
	
	\begin{block}{Sintaxis}
		La sintaxis para este tipo de comandos es \texttt{\textbackslash Comando<Rango>\{Contenido\}},  donde \texttt{Rango} especifica en qué pasos de la animación se tiene el efecto, por ejemplo:
			\begin{itemize}
				\item \texttt{2-} $\rightarrow$ Del paso 2 en adelante.
				\item \texttt{-3} $\rightarrow$ Hasta el paso 3 (inclusive).
				\item \texttt{2-5} $\rightarrow$ Del paso 2 al 5 (ambos inclusive).
				\item \texttt{1-3,5} $\rightarrow$ Del paso1 al 3 y en el 5.
			\end{itemize}
	\end{block}
\end{frame}

	%% El comando \texttt{\textbackslash onslide}  
\subsection{\texttt{El comando \textbackslash onslide}}
\begin{frame}
	\frametitle{\secname : \subsecname}
	
	Este comando determina el orden de los elementos, pero hace que estos \alert{sigan ocupando espacio}. Por ejemplo
	
	\begin{block}
		\onslide<1->{Este texto se muestra desde el primer paso}	\\
		\onslide<2>{Este en el paso 2}						\\
		\onslide<2-3>{Y este en los pasos 2 y 3}				\\
		\onslide<4->{Este se muestra a partir del 4}
	\end{block}

\end{frame}

	%% El comando \texttt{\textbackslash only} 
\subsection{El comando \texttt{\textbackslash only}}
\begin{frame}
	\frametitle{\secname : \subsecname}
	
	Este comando funciona de la misma forma que \texttt{\textbackslash onslide} pero con la diferencia que aquí el texto no mostrado \alert{no ocupa espacio}. Este comando puede servir para hacer sustituciones.
	
	\begin{block}
		\onslide<1->{Este texto se muestra desde el primer paso}	\\
		\only<2>{Este solo se muestra en el paso 2}
		\onslide<3->{Este desde  el 3 en adelante}	\\
		\onslide<4>{Esto en el cuarto}
	\end{block}

\end{frame}

	%% Efectos de animación incrementales %%
\subsection{Efectos de animación incrementales}
\begin{frame}
	\frametitle{\secname : \subsecname}
	
	Podemos usar efectos de animación usando rangos dentro de los mismos entornos \texttt{itemize} y \texttt{enumerate}. La sintaxis es la siguiente:
		\begin{block}{}
		\begin{itemize}
			\item<1-> Primer item.
			\item<2-> Segundo item.
			\item<3-> Tercer item.
		\end{itemize}
		\end{block}
\end{frame}

\begin{frame}
	\frametitle{\secname : \subsecname}
	
	Sin embargo, hay una manera equivalente para ese orden en específico:
		\begin{block}{}
		\begin{itemize}[<+->]
			\item Primer item.
			\item Segundo item.
			\item Tercer item.
		\end{itemize}
		\end{block}
\end{frame}

\begin{frame}
	\setbeamercovered{transparent}
	
	\frametitle{\secname : \subsecname}
	
	Tal vez nos gustaría que el elemento activo aparezca resaltado:
		\begin{block}{}
		\begin{itemize}[<+- | alert@+>]
			\item Primer item.
			\item Segundo item.
			\item Tercer item.
		\end{itemize}
		\end{block}
	
	También es posible este efecto en el uso de rangos, con \texttt{\textbackslash alert<Rango>\{Texto\}}.
\end{frame}

\begin{frame}
	\frametitle{\secname : \subsecname}
	
	\begin{block}{}
	Es posible usar, en lugar de \texttt{alert} otros comandos como
		\begin{itemize}
			\item \texttt{uncover}: es equivalente a \texttt{onslida} y está por defecto.
			\item \texttt{only}: el mismo que vimos antes.
			\item \texttt{visible}: Texto invisible si no es mostrado.
		\end{itemize}
	\end{block}
	
	Por ejemplo
		\begin{itemize}[<+- | only@+>]
			\item Primer item.
			\item Segundo item.
			\item Tercer item.
		\end{itemize}
\end{frame}

	%% Otros comandos con efectos de animación %%
\begin{frame}
	\frametitle{\secname : \subsecname}
	
	\begin{block}{}
	Los siguientes comandos admiten efectos de animación
		\begin{itemize}
			\item \texttt{\textbackslash textbf<Rango>\{Texto\}}
			\item \texttt{\textbackslash color<Rango>\{Color\}} (o \texttt{\textbackslash textcolor<Rango>\{Color\}\{Texto\}})
			\item \texttt{\textbackslash includegraphics<Rango>[Opciones]\{archivo\}}
		\end{itemize}
	\end{block}
	
	Como por ejemplo
		\begin{itemize}
			\item<1-> \textbf<1>{Primer item.}
			\item<2-> \textcolor<2>{blue}{Segundo item.}
			\item<3-> Tercer item \includegraphics<4>[height = 4mm]{tick.png}
		\end{itemize}
	
\end{frame}

\end{document}

