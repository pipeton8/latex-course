% Comencemos con el mismo preámbulo anterior
\documentclass{article}

\usepackage[utf8]{inputenc} 
\usepackage[spanish]{babel}
\usepackage{mathtools, amsfonts, amsmath, amsthm, amssymb}

% Como podemos ver, el texto de muestra no tiene los mismos márgenes
% que el texto que escribimos antes, significa entonces que queremos
% cambiar los márgenes y para eso usamos el siguiente paquete:
\usepackage{setspace}

% Los márgenes de la página se ajustan fácilmente usando el paquete geometry.
% El tamaño ``2cm'' puede ajustarse y también puede cambiar de unidades (in en lugar de cm, por ejemplo).
\usepackage[margin=2cm]{geometry}

% El interlineado se ajusta con el comando spacing. El valor 1.2 usualmente queda bien, pero
% pueden jugar con otros valores.
\spacing{1.2}

% Ahora, si vemos el texto, podemos notar que todo el teorema tiene la letra en itálicas (o cursiva), a diferencia del texto de la demostración. Esto es porque el teorema se encuentra dentro de un ``entorno''. Los entornos permiten escribir partes del texto de una manera particular y con ciertas configuraciones. En este caso, el entorno en el que se encuentra el teorema, escribe el texto en cursiva y enumera el teorema. Para crear un entorno de teorema nuevo escribimos:
\newtheorem{teo}{Teorema}

% Esto genera un entorno de teorema, cuyo título es teorema y se abrevia teo.
% Para llamar a un entorno en el texto, escribimos 
%	\begin{nombre entorno}
%		aquí va el texto
%	\end{nombre entorno}

% Hay distintos entornos en LaTeX. Algunos son ``center'', ``align'', ``enumerate'', ``proof'', etc.
% Los entornos son uno de los elementos más usados en LaTeX.

% Ahora, intentemos reproducir el texto de muestra. Comenzamos con
\begin{document}

% Ahora, queremos escribir el teorema en el entorno que escribimos, para ello iniciamos el entorno como se mencionó anteriormente:
\begin{teo}
% En la primera linea vemos ya un texto matemático. Cuando estos están en la misma linea 
% de un texto se escriben entre signos $, como en la siguiente linea. 
Demuestre que si $1 \leq p \leq q \leq \infty$, entonces

% El siguiente texto matemático está centrado y no en la misma linea del texto. 
% En estos casos hay muchas formas de escribir, pero la más usual es similar a la anterior
% salvo que usamos dos $ en lugar de uno.
	$$\|x\|_{\ell^{q}} \leq \|x\|_{\ell^{p}}$$
	
En particular $\ell^{p} \subset \ell^{q}$.

%Terminado el teorema, cerramos el entorno.
\end{teo}

% La demostración también es un entorno. Su trabajo es poner ``Demostración'' al comienzo
% y el símbolo (por defecto un cuadrado blanco) al final. Se llama igual que todos los entornos:
\begin{proof} El caso $q = \infty$ es claro:

% LaTeX es un poco tonto con los paréntesis. No sabe detectar automáticamente cuando algo es
% demasiado grande como para agrandar los paréntesis. Para ello usamos \left y \right. Estos comandos
% sirven para ajustar los paréntesis de cualquier tipo al tamaño de lo que contienen 
	$$\|x\|_{\ell^{\infty}} 
		= 	\sup_{j \in \mathbb{N}} |x_{j}|
		= 	\left(\sup_{j \in \mathbb{N}} |x_{j}|^{p}\right)^{1/p} 
		\leq	\left(\sum_{j \in \mathbb{N}} |x_{j}|^{p}\right)^{1/p}
		=	\|x\|_{\ell^{p}}
	$$

Además, si $p = q$ el resultado es directo. Consideremos entonces $1 \leq p < q < \infty$, tenemos que
% El siguiente entorno también es muy útil, permite alinear las fórmulas matemáticas respecto 
% a un cierto elemento. El asterisco, muy usado en LaTeX, hace que la fórmula no se enumere.
% Para alinear, usamos el símbolo & y para saltar de una linea a la siguiente usamos \\
	\begin{align*}
		\|x\|_{\ell^{q}}^{q}
			&=		\sum_{j \in \mathbb{N}} |x_{j}|^{q}	\\
			&=		\sum_{j \in \mathbb{N}} |x_{j}|^{p} |x_{j}|^{q-p}	\\
			&\leq		\|x\|^{q-p}_{\ell^{\infty}} \sum_{j \in \mathbb{N}} |x_{j}|^{p}	\\
			&=		\|x\|^{q-p}_{\ell^{\infty}} \|x\|_{\ell^{p}}^{p}	\\
			&\leq		\|x\|^{q-p}_{\ell^{p}} \|x\|_{\ell^{p}}^{p}	\\
			&=		\|x\|_{\ell^{p}}^{q}		 
	\end{align*}

De donde se obtiene el resultado.

% Para terminar, cerramos el entorno.
\end{proof}                                          

% Y terminamos el documento.                   
\end{document}

