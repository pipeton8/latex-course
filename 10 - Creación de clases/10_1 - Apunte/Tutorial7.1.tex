%%Preámbulo
\RequirePackage[patch]{kvoptions}

% Tipo de documento %
\documentclass[	docname= Sistemas\ Din\'amicos,
				finished=1,
				semester=1,
				year=2017,
				professor=Godofredo\ Iommi,
				sigla=MAT2565]{apunte}

% Paquetes %
\usepackage{tutorialLaTeX}

% Otras tonteras %
\graphicspath{{./Graphics/}}

%%% Documento %%%
\begin{document}

% Portada %
\dotitlepage

% Índice %
\doindex
\tableofcontents
\newpage

\dobody

%%%	Introducción	%%%
\section{Introducción}

	%%	Definiciones	%%
\subsection{Definiciones}
\begin{defn}[Sistema dinámico] Un sistema dinámico corresponde a un par $(X, f)$, donde $X$ es un conjunto y $f: X \to X$ es una función.
\end{defn}

\begin{defn} Sea $f: X \to X$ una función. Diremos que $A \subset X$ es
	\begin{enumerate}[\indent 1)]
		\item \rojo{$f$-invariante} si
				$$f^{-1}(A) = \{x \in X : f(x) \in A\} = A$$
		
		\item \rojo{Positivamente $f$-invariante} si $f(A) \subset A$.
		\item \rojo{Negativamente $f$-invariante} si $f^{-1}(A) \subset A$.
	\end{enumerate}
\end{defn}

\begin{defn} Sea $x \in X$, la \rojo{órbita} del punto $x$ es el conjunto $\{f^{n}(x) : n \in \N \cup \{0\} \}$, donde
	$$f^{n}(x) := \underbrace{(f \circ  \cdots \circ f)}_{n\text{-veces}}(x)$$

Si además $f$ es invertible, entonces podemos definir órbita de $x$ por $\{f^{n}(x): n \in \Z\}$.
\end{defn}

\begin{defn} Diremos que $x \in X$ es \rojo{punto periódico} de $f$ si existe $N \in \N$ tal que
	$$f^{N}(x) = x$$

Al menor $N$ con tal propiedad le llamaremos \rojo{periodo minimal} de $x$.
\end{defn}

\begin{ex} Si $f(x) = x$, entonces todos $x$ es un punto periódico de periodo minimal igual a 1. Observemos que para cada $n \in \N$, se tiene $f^{n}(x) = x$.
\end{ex}

\end{document}