% Consideremos el código de la sección 4 y mejorémoslo
\documentclass{article}

% Paquetes %
\usepackage{setspace}
\usepackage{tutorialLaTeX}	% Incorporamos nuestro nuevo paquete con la sintaxis de paquetes usual.

% Márgenes %
\oddsidemargin = 0.3 cm
\textwidth = 17 cm
\textheight = 24 cm
\headsep = 0.8 cm
\hoffset = -1 cm
\voffset = -2 cm
\spacing{1.4}

% Está un poco mejor, pero no tanto. Todavía molestan los márgenes y el texto de la portada aún ocupa mucho espacio. Eso lo mejoraremos con una clase.

% Documento %
\begin{document}

\begin{titlepage}
	\begin{minipage}{2.5cm}
		\includegraphics[width=2cm]{logouccolor.png}
	\end{minipage}
	\begin{minipage}{13 cm}
		\begin{flushleft}
   			\noindent\large{\sc
				Pontificia Universidad Católica de Chile \\ 
		     		Facultad de Matemáticas \\ 
		     		Departamento de Matemáticas \\ 
				Segundo semestre 2017
		     	}
		\end{flushleft}
	\end{minipage}
	
	% Título %
\begin{center}
	\vspace*{\fill}
		\Huge\textbf{Primera portada}	\\
		\Huge\textbf{taller de \LaTeX}	\\
		\LARGE{Yo mismo} \\
	\vspace*{\fill}
		
	\vfill
		
	\Large{Profesor: No importa por ahora}
\end{center}
\end{titlepage}

\begin{teo} Demuestre que si $1 \leq p \leq q \leq \infty$, entonces
	$$\|x\|_{\ell^{q}} \leq \|x\|_{\ell^{p}}$$
	
En particular $\ell^{p} \subset \ell^{q}$.
\end{teo}

\begin{proof} El caso $q = \infty$ es claro:
	$$\|x\|_{\ell^{\infty}} 
		= 	\sup_{j \in \N} |x_{j}|
		= 	\left(\sup_{j \in \N} |x_{j}|^{p}\right)^{1/p} 
		\leq	\left(\sum_{j \in \N} |x_{j}|^{p}\right)^{1/p} = \|x\|_{\ell^{p}}
	$$

Además, si $p = q$ el resultado es directo. Consideremos entonces $1 \leq p < q < \infty$, tenemos que
	\begin{align*}
		\|x\|_{\ell^{q}}^{q}
			&=		\sum_{j \in \N} |x_{j}|^{q}	\\
			&=		\sum_{j \in \N} |x_{j}|^{p} |x_{j}|^{q-p}	\\
			&\leq		\|x\|^{q-p}_{\ell^{\infty}} \sum_{j \in \N} |x_{j}|^{p}	\\
			&=		\|x\|^{q-p}_{\ell^{\infty}} \|x\|_{\ell^{p}}^{p}	\\
			&\leq		\|x\|^{q-p}_{\ell^{p}} \|x\|_{\ell^{p}}^{p}	\\
			&=		\|x\|_{\ell^{p}}^{q}		 
	\end{align*}

De donde se obtiene el resultado.
\end{proof}                                          

\begin{obst} Hemos visto que $x = \{x_{n}\}_{n \in \N}$ definida por
	$$x_{n} = \twodef	{\frac{1}{n-1}}{n > 1}
					{0}{n = 1}$$

pertenece a $\ell^{2}$, por lo que por el teorema anterior, pertenece a $\ell^{q}$ para todo $q \geq 2$.
\end{obst}
\end{document}

