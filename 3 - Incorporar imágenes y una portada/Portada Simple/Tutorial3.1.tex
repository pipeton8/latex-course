% Consideremos el código del teorema anterior e intentemos incorporarle una portada
\documentclass{article}

% Paquetes %
\usepackage[utf8]{inputenc} 
\usepackage[spanish]{babel}
\usepackage{mathtools, amsfonts, amsmath, amsthm, amssymb}
\usepackage{setspace}

% Márgenes %
\oddsidemargin = 0.3 cm
\textwidth = 17 cm
\textheight = 24 cm
\headsep = 0.8 cm
\hoffset = -1 cm
\voffset = -2 cm
\spacing{1.4}

% Entornos %
\newtheorem{teo}{Teorema}

% Una forma de crear una portada es usando un comando que viene incorporado en LaTeX
% Esta portada es muy básica y, típicamente, no muy presentable. Pero es una forma fácil de entregarle
% una portada a un texto que podría no ser tan importante. Esta puede mejorarse con más elementos
% pero no nos preocuparemos de esto por ahora.

% Primero debemos decirle a LaTeX cual es el título, quien es el autor y, opcionalmente,
% cual es la fecha. Si esta última no se especifica, LaTeX asume la fecha del día en que se compila
% el texto.
\title{Primera portada del taller de \LaTeX}
\author{Yo mismo}

% Documento %
\begin{document}
% Ahora, debemos decirle a LaTeX que haga la portada. Para ello usaremos el entorno titlepage, que deja
% la portada en una página aparte, y el comando \maketitle, que usa los datos entregados anteriormente y
% los usa para crear la portada.

\begin{titlepage}
\thispagestyle{empty}
\maketitle
\end{titlepage}

\begin{teo} Demuestre que si $1 \leq p \leq q \leq \infty$, entonces
	$$\|x\|_{\ell^{q}} \leq \|x\|_{\ell^{p}}$$
	
En particular $\ell^{p} \subset \ell^{q}$.
\end{teo}

\begin{proof} El caso $q = \infty$ es claro:
	$$\|x\|_{\ell^{\infty}} 
		= 	\sup_{j \in \mathbb{N}} |x_{j}|
		= 	\left(\sup_{j \in \mathbb{N}} |x_{j}|^{p}\right)^{1/p} 
		\leq	\left(\sum_{j \in \mathbb{N}} |x_{j}|^{p}\right)^{1/p} = \|x\|_{\ell^{p}}
	$$

Además, si $p = q$ el resultado es directo. Consideremos entonces $1 \leq p < q < \infty$, tenemos que
	\begin{align*}
		\|x\|_{\ell^{q}}^{q}
			&=		\sum_{j \in \mathbb{N}} |x_{j}|^{q}	\\
			&=		\sum_{j \in \mathbb{N}} |x_{j}|^{p} |x_{j}|^{q-p}	\\
			&\leq		\|x\|^{q-p}_{\ell^{\infty}} \sum_{j \in \mathbb{N}} |x_{j}|^{p}	\\
			&=		\|x\|^{q-p}_{\ell^{\infty}} \|x\|_{\ell^{p}}^{p}	\\
			&\leq		\|x\|^{q-p}_{\ell^{p}} \|x\|_{\ell^{p}}^{p}	\\
			&=		\|x\|_{\ell^{p}}^{q}		 
	\end{align*}

De donde se obtiene el resultado.
\end{proof}                                          

\end{document}

