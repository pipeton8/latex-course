% Consideremos el código del teorema anterior e intentemos incorporarle una portada
\documentclass{article}

% Paquetes %
\usepackage[utf8]{inputenc} 
\usepackage[spanish]{babel}
\usepackage{mathtools, amsfonts, amsmath, amsthm, amssymb}
\usepackage{setspace}

% En esta portada usaremos una imagen, por lo que necesitamos usar el paquete
% encargado de lidiar con ellas
\usepackage{graphicx}

% Márgenes y espaciado %
\usepackage[margin=2cm]{geometry}
\spacing{1.2}

% Entornos %
\newtheorem{teo}{Teorema}

% Documento %
\begin{document}

% La segunda forma es creando una portada a mano, ubicando los elementos a nuestro gusto.
% Seguiremos usando el entorno titlepage, pero distribuiremos los elementos de portada 
% según como nos plazca
\begin{titlepage}

% Este entorno minipage permite ubicar correctamente la imagen, separada del texto que la sigue, para que
% podamos controlar su tamaño independiente del texto
	\begin{minipage}{2.5cm}
		\includegraphics[width=2cm]{logouccolor.png}
	\end{minipage}
	
% Por su parte, esta minipage controla el texto que está al lado de la imagen, para controlar
% su tamaño y ubicación respecto a la imagen de manera más simple
	\begin{minipage}{13 cm}
	
	% El siguiente entorno alinea el texto a la izquierda (flushright lo alinea a la derecha)
		\begin{flushleft}
   			\noindent\large{\sc
				Pontificia Universidad Católica de Chile \\ 
		     		Facultad de Matemáticas \\ 
		     		Departamento de Matemáticas \\ 
				Segundo semestre 2017
		     	}
		\end{flushleft}
	\end{minipage}
	
	% Título %
\begin{center}
	\vspace*{\fill} % Este comando, junto con el mismo más abajo asegura que el texto encerrado queda en el centro de la página
		\Huge\textbf{Primera portada}	\\
		\Huge\textbf{taller de \LaTeX}	\\
		\LARGE{Yo mismo} \\
	\vspace*{\fill}
		
	\vfill % Este comando asegura que el siguiente texto queda en el final de la página
		
	\Large{Profesor: Felipe del Canto}

\end{center}
\end{titlepage}

\begin{teo} Demuestre que si $1 \leq p \leq q \leq \infty$, entonces
	$$\|x\|_{\ell^{q}} \leq \|x\|_{\ell^{p}}$$
	
En particular $\ell^{p} \subset \ell^{q}$.
\end{teo}

\begin{proof} El caso $q = \infty$ es claro:
	$$\|x\|_{\ell^{\infty}} 
		= 	\sup_{j \in \mathbb{N}} |x_{j}|
		= 	\left(\sup_{j \in \mathbb{N}} |x_{j}|^{p}\right)^{1/p} 
		\leq	\left(\sum_{j \in \mathbb{N}} |x_{j}|^{p}\right)^{1/p} = \|x\|_{\ell^{p}}
	$$

Además, si $p = q$ el resultado es directo. Consideremos entonces $1 \leq p < q < \infty$, tenemos que
	\begin{align*}
		\|x\|_{\ell^{q}}^{q}
			&=		\sum_{j \in \mathbb{N}} |x_{j}|^{q}	\\
			&=		\sum_{j \in \mathbb{N}} |x_{j}|^{p} |x_{j}|^{q-p}	\\
			&\leq		\|x\|^{q-p}_{\ell^{\infty}} \sum_{j \in \mathbb{N}} |x_{j}|^{p}	\\
			&=		\|x\|^{q-p}_{\ell^{\infty}} \|x\|_{\ell^{p}}^{p}	\\
			&\leq		\|x\|^{q-p}_{\ell^{p}} \|x\|_{\ell^{p}}^{p}	\\
			&=		\|x\|_{\ell^{p}}^{q}		 
	\end{align*}

De donde se obtiene el resultado.
\end{proof}                                          

\end{document}

